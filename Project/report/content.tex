\section{Introduction}

The $n$-queens problem, a combinatorial search problem, concerns the non-attacking (horizontally, vertically, and diagonally) placement of $n$ queens on a $n\times n$ chessboard. The $n$-queens and similar Constraint Satisfaction Problems (CSPs) are classical examples of the limitations of simple backtracking search, with an exponential worst case performance that renders solving for large $n$ impractical \citep{sosic90, backtracking}. Although many efficient heuristics have already been proposed for this problem \citep{sosic91, hu03, aima, engel07, agarwal12}, it still is a popular test bed for new Artificial Intelligence (AI) search problem methods. Whilst a toy problem per se, it has found some practical applications such as VLSI routing and testing, data compression, maximum full range communication and parallel optical computing \citep{sosic91, hu03}.

This problem has (at least) two variants depending on the desired number of solutions. A single solution can actually be found trivially without search, since explicit solutions exist $\forall \ n \ge 4$ \citep{trivial}. On the other hand, finding all possible solutions is non-trivial. In this project we will focus on the former variant, implementing and comparing the performance of two different search algorithms: a local search algorithm, and a constraint satisfaction method. We will first describe the problem mathematically and define the performance indicators for our comparison. In Section \ref{sec:methods} we will describe the implemented algorithms, and we will subsequently illustrate and discuss their performance.

\section{Problem Formulation}

\subsection{Mathematical Model}



\subsection{Performance Indicators}


\section{Methods}
\label{sec:methods}

\subsection{Min-Conflicts Algorithm}

\subsection{Forward Checking with Most Restricted Value Algorithm}

\section{Results}

\section{Conclusions}












